% (c) 2002 Matthew Boedicker <mboedick@mboedick.org> (original author) http://mboedick.org
% (c) 2003-2007 David J. Grant <davidgrant-at-gmail.com> http://www.davidgrant.ca
% (c) 2008 Nathaniel Johnston <nathaniel@nathanieljohnston.com> http://www.nathanieljohnston.com
%
% (c) 2012 Scott Clark <sc932@cornell.edu> cam.cornell.edu/~sc932
%
%This work is licensed under the Creative Commons Attribution-Noncommercial-Share Alike 2.5 License. To view a copy of this license, visit http://creativecommons.org/licenses/by-nc-sa/2.5/ or send a letter to Creative Commons, 543 Howard Street, 5th Floor, San Francisco, California, 94105, USA.

\documentclass[letterpaper,11pt]{article}
\newlength{\outerbordwidth}
\pagestyle{empty}
\raggedbottom
\raggedright
\usepackage[svgnames]{xcolor}
\usepackage{framed}
\usepackage{hyperref}
\usepackage{tocloft}
\definecolor{dgray}{gray}{0.4}
\hypersetup{
  colorlinks=true,
  urlcolor=dgray
}

%-----------------------------------------------------------
%Edit these values as you see fit

\setlength{\outerbordwidth}{3pt}  % Width of border outside of title bars
\definecolor{shadecolor}{gray}{0.75}  % Outer background color of title bars (0 = black, 1 = white)
\definecolor{shadecolorB}{gray}{0.93}  % Inner background color of title bars


%-----------------------------------------------------------
%Margin setup

\setlength{\evensidemargin}{-0.25in}
\setlength{\headheight}{0in}
\setlength{\headsep}{0in}
\setlength{\oddsidemargin}{-0.25in}
\setlength{\paperheight}{11in}
\setlength{\paperwidth}{8.5in}
\setlength{\tabcolsep}{0in}
\setlength{\textheight}{9.5in}
\setlength{\textwidth}{7in}
\setlength{\topmargin}{-0.3in}
\setlength{\topskip}{0in}
\setlength{\voffset}{0.1in}


%-----------------------------------------------------------
%Custom commands
\newcommand{\resitem}[1]{\item #1 \vspace{-2pt}}
\newcommand{\resheading}[1]{\vspace{8pt}
  \parbox{\textwidth}{\setlength{\FrameSep}{\outerbordwidth}
    \begin{shaded}
\setlength{\fboxsep}{0pt}\framebox[\textwidth][l]{\setlength{\fboxsep}{4pt}\fcolorbox{shadecolorB}{shadecolorB}{\textbf{\sffamily{\mbox{~}\makebox[6.762in][l]{\large #1} \vphantom{p\^{E}}}}}}
    \end{shaded}
  }\vspace{-5pt}
}
\newcommand{\ressubheading}[4]{
\begin{tabular*}{6.5in}{l@{\extracolsep{\fill}}r}
		\textbf{#1} & #2 \\
		\textit{#3} & \textit{#4} \\
\end{tabular*}\vspace{-6pt}}
%-----------------------------------------------------------


\begin{document}

\begin{tabular*}{7in}{l@{\extracolsep{\fill}}r}
\textbf{\Large Kunal Singhal}  
& Ph: +91 9971490655 \\ 
Junior Undergrad, Computer Science,
& \href{mailto:kunal.cs112@cse.iitd.ernet.in}{kunal.cs112@cse.iitd.ernet.in}\\
Indian Institute of Technology Delhi
& \href{http://www.cse.iitd.ernet.in/~cs1120231}{http://www.cse.iitd.ernet.in/$\sim$cs1120231} \\
\end{tabular*}
\\


%%%%%%%%%%%%%%%%%%%%%%%%%%%%%%
\resheading{Education}
%%%%%%%%%%%%%%%%%%%%%%%%%%%%%%
\ressubheading{Indian Institute of Technology Delhi}{New Delhi, India}{Bachelor of Technology}{2012 - 2016 (expected)}

\ \\
CGPA: 9.442


%%%%%%%%%%%%%%%%%%%%%%%%%%%%%%
\resheading{Relevant Courses Taken}
%%%%%%%%%%%%%%%%%%%%%%%%%%%%%%
\begin{tabbing}
Artificial Intelligence \hspace{1.5cm}\= Design Practices in Computer Science \hspace{1cm}\= Digital Electronic Circuits \\
Data Structures \> Discrete Mathematical Structures \> Combinatorics \\
Calculus and Analysis \> Linear Algebra and Matrices \> Quantum Physics \\ 
Programming Languages\ \> Cryptography\> Computer Architecture\\
Probability Theory \> Stochastic Processes \> Microeconomics\\ 
Algorithms\> Systems Biology \> Natural Language Processing \\
Graph Algorithms \> Machine Learning*\\ 
\end{tabbing}
\vspace{-0.5cm}
\hspace{12.5cm} \footnotesize{
*online course at coursera.org
}

%%%%%%%%%%%%%%%%%%%%%%%%%%%%%%
\resheading{Projects}
%%%%%%%%%%%%%%%%%%%%%%%%%%%%%%

\begin{itemize}

\item 
\ressubheading{Classification of tweets based on Personally Identifiable Information}{UC Irvine, CA}{Reseach intern with ICS department under Prof. Sharad Mehrotra}{Summer 2014 - present}
\begin{itemize}
\item rule based model of a PII learnt
\item a human analyst used for learning the model
\item active learning incorporated to use human resource efficiently
\end{itemize}

\item
\ressubheading{Next generation Open Information Extraction}{IIT Delhi}{proposed for SURA under Prof. Mausam}{Summer 2014 - present}
\begin{itemize}
\item list identification and subsequent division to improve recall
\item number and conjunction identification and extracting special relations to improve precision and recall
\end{itemize}

\item
\ressubheading{AI Game Player for Connect n,m,k}{IIT Delhi}{course project under Prof. Mausam}{Jan 2014 - Apr 2014}
\begin{itemize}
\item modified version of connect 4 game
\item the player uses mini max with alpha beta pruning
\item mini-max was testing against other techniques such as UCB and UCT
\end{itemize}


\item
\ressubheading{Prolog Interpreter in OCaml}{IIT Delhi}{course project under Prof. Sanjeeva Prasad}{Feb 2014 - Mar 2014}
\begin{itemize}
\item ocamllex and ocamlyacc used for lexing and parsing respectively 
\item ideas of backtracking and unification of terms were used to implement the relational backbone of Prolog interpreter
\end{itemize}

\item
\ressubheading{3D bike race game}{IIT Delhi}{course project under Prof. Subodh Kumar}{July 2013 - Sept 2013­}
\begin{itemize}
\item used OpenGL as graphics library
\item a dedicated physics engine was programmed
\item a database for High Scores is maintained
\item used frustum culling for enhancing the game speed
\end{itemize}


\item
\ressubheading{Single Cycle Processor Design}{IIT Delhi}{course project under Prof. Smruti Ranjan Sarangi}{Nov 2013  - Jan 2014­}
\begin{itemize}
\item a single cycle risc processor was designed using
\item the design implements simpleRisc ISA which is much similar to ARM ISA
\end{itemize}


\item
\ressubheading{Social Network Simulation and Analysis}{IIT Delhi}{course project under Prof. Subodh Kumar}{Sep 2013  - Nov 2013}
\begin{itemize}
\item multiprocessing and multithreading used to achieve simulation.
\item inter process communication is established by Message Queues
\item final network after the simulation is stored as a graphml file.
\item in the analysis of the network, various queries such as shortest path, importance and clique size can be performed.
\end{itemize}

\end{itemize}
%%%%%%%%%%%%%%%%%%%%%%%%%%%%%%
\resheading{Sport Programming}
%%%%%%%%%%%%%%%%%%%%%%%%%%%%%%

\begin{itemize}

\item Indian team member in \textbf{International Olympiad in Informatics, 2012}
\item My team (team Angle) \textbf{ranked 9th in ICPC Amritapuri Regionals.} Best in IIT Delhi. 
\item TopCoder: Yellow Rated - 1758 (among top 20 in India) 
\end{itemize}

%%%%%%%%%%%%%%%%%%%%%%%%%%%%%%
\resheading{Awards, Grants \& Honours}
%%%%%%%%%%%%%%%%%%%%%%%%%%%%%%
\begin{tabular*}{7in}{l@{\extracolsep{\fill}}r}
 \textbf{International Olympiad in Informatics 2012}  & \textsc{July 2012} \\
Indian Team Member \\\\
 \textbf{International Physics Olympiad 2012}  & \textsc{July 2012} \\
Silver Award\\\\
 \textbf{Asian Physics Olympiad 2012} & \textsc{May 2012} \\
Silver Award\\\\
 \textbf{Indian National Mathematics Olympiad 2011} & \textsc{2011} \\
All India First Position \\\\
 \textbf{Institute Silver Medal for Academic Excellenece } & \textsc{Feb 2013} \\
Given for obtaining highest CGPA at IIT Delhi\\\\
 \textbf{IIT-JEE 2012 All India Rank 18} & \textsc{May 2012} \\
Got 18th rank among more than 500,000 students\\\\
 \textbf{Aditya Birla Scholarship 2012} & \textsc{Sep 2012} \\
Given only to 15 engineering students all India\\\\
 \textbf{OP Jindal Engineering and Management Scholarship 2012} & \textsc{Sep 2012} \\
Given to one student of each Year\\\\
 \textbf{KVPY Scholarship 2011} & \textsc{2011} \\
Scholarship given for encouraging science students \\\\
 \textbf{AIEEE 2012 All India Rank 19} & \textsc{May 2012} \\
Got 19th position among over 1 million students.
\end{tabular*}


\clearpage
%%%%%%%%%%%%%%%%%%%%%%%%%%%%%%
\resheading{Designing and Coding Skills}
%%%%%%%%%%%%%%%%%%%%%%%%%%%%%%

\begin{tabular*}{5in}{l@{\extracolsep{\fill}}l}
{\bf Extensive}
& \textsc{python}, C++\\[2pt]
{\bf Intermediate}
& \textsc{html5}, \textsc{css3}, \textsc{JavaScript, C, php}, \LaTeX, \textsc{Java, sql, bash, numpy}\\[2pt]
{\bf Basic}
& \textsc{Django Framework, Apple Script, Standard ML, BASIC, Perl} \\[-5pt]
\end{tabular*}

%%%%%%%%%%%%%%%%%%%%%%%%%%%%%%
\resheading{Community Involvement}
%%%%%%%%%%%%%%%%%%%%%%%%%%%%%%

\begin{tabular*}{7in}{l@{\extracolsep{\fill}}r}
Lecturer and Operations Coordinator, Coding Club & July, 2014 - Present \\ [2pt]
Cultural Secretary, ACES & Dec, 2012 - Apr, 2014 \\ [2pt]
Convener from Department, AIC & Mar, 2013 - Mar, 2014 
\end{tabular*}

%%%%%%%%%%%%%%%%%%%%%%%%%%%%%%
\resheading{Other Interests}
%%%%%%%%%%%%%%%%%%%%%%%%%%%%%%

I dance a lot. I love to play air guitar. And I just cannot resist talking philosophy. I also have a blog named "Tenet". 

\end{document}