% (c) 2002 Matthew Boedicker <mboedick@mboedick.org> (original author) http://mboedick.org
% (c) 2003-2007 David J. Grant <davidgrant-at-gmail.com> http://www.davidgrant.ca
% (c) 2008 Nathaniel Johnston <nathaniel@nathanieljohnston.com> http://www.nathanieljohnston.com
%
% (c) 2012 Scott Clark <sc932@cornell.edu> cam.cornell.edu/~sc932
%
%This work is licensed under the Creative Commons Attribution-Noncommercial-Share Alike 2.5 License. To view a copy of this license, visit http://creativecommons.org/licenses/by-nc-sa/2.5/ or send a letter to Creative Commons, 543 Howard Street, 5th Floor, San Francisco, California, 94105, USA.

\documentclass[letterpaper,11pt]{article}
\newlength{\outerbordwidth}
\pagestyle{empty}
\raggedbottom
\raggedright
\usepackage[svgnames]{xcolor}
\usepackage{framed}
\usepackage{hyperref}
\usepackage{tocloft}
\definecolor{dgray}{gray}{0.4}
\hypersetup{
  colorlinks=true,
  urlcolor=dgray
}

%-----------------------------------------------------------
%Edit these values as you see fit

\setlength{\outerbordwidth}{3pt}  % Width of border outside of title bars
\definecolor{shadecolor}{gray}{0.75}  % Outer background color of title bars (0 = black, 1 = white)
\definecolor{shadecolorB}{gray}{0.93}  % Inner background color of title bars


%-----------------------------------------------------------
%Margin setup

\setlength{\evensidemargin}{-0.25in}
\setlength{\headheight}{0in}
\setlength{\headsep}{0in}
\setlength{\oddsidemargin}{-0.25in}
\setlength{\paperheight}{11in}
\setlength{\paperwidth}{8.5in}
\setlength{\tabcolsep}{0in}
\setlength{\textheight}{9.5in}
\setlength{\textwidth}{7in}
\setlength{\topmargin}{-0.3in}
\setlength{\topskip}{0in}
\setlength{\voffset}{0.1in}


%-----------------------------------------------------------
%Custom commands
\newcommand{\resitem}[1]{\item #1 \vspace{-2pt}}
\newcommand{\resheading}[1]{\vspace{8pt}
  \parbox{\textwidth}{\setlength{\FrameSep}{\outerbordwidth}
    \begin{shaded}
\setlength{\fboxsep}{0pt}\framebox[\textwidth][l]{\setlength{\fboxsep}{4pt}\fcolorbox{shadecolorB}{shadecolorB}{\textbf{\sffamily{\mbox{~}\makebox[6.762in][l]{\large #1} \vphantom{p\^{E}}}}}}
    \end{shaded}
  }\vspace{-5pt}
}
\newcommand{\ressubheading}[4]{
\begin{tabular*}{6.5in}{l@{\cftdotfill{\cftsecdotsep}\extracolsep{\fill}}r}
		\textbf{#1} & #2 \\
		\textit{#3} & \textit{#4} \\
\end{tabular*}\vspace{-6pt}}
%-----------------------------------------------------------


\begin{document}

\begin{tabular*}{7in}{l@{\extracolsep{\fill}}r}
\textbf{\Large Kunal Singhal}  
& Ph: +91 9971490655 \\ 
UG, Computer Science and Engineering
& \href{mailto:kunal.cs112@cse.iitd.ernet.in}{kunal.cs112@cse.iitd.ernet.in}\\
Indian Institute of Technology Delhi
& \href{http://www.cse.iitd.ernet.in/~cs1120231}{http://www.cse.iitd.ernet.in/$\sim$cs1120231} \\
\end{tabular*}
\\


%%%%%%%%%%%%%%%%%%%%%%%%%%%%%%
\resheading{Education}
%%%%%%%%%%%%%%%%%%%%%%%%%%%%%%
\ressubheading{Indian Institute of Technology Delhi}{New Delhi, India}{Bachelor of Technology}{2012 - 2016 (expected)}

\ \\
CGPA: 9.56


%%%%%%%%%%%%%%%%%%%%%%%%%%%%%%
\resheading{Relavent Courses Taken}
%%%%%%%%%%%%%%%%%%%%%%%%%%%%%%
\begin{tabbing}
Computer Architecture \hspace{1.5cm}\= Design Practices in Computer Sciecnce \hspace{1cm}\= Digital Electronic Circuits \\
Data Structures \> Descrete Mathematical Structures \> Combinatorics \\
Calculas and Analysis \> Linear Algebra and Matrices \> Quantum Physics\textsuperscript{*} \\ 
Programming Languages\textsuperscript{*} \> Numerical and Scientific Computation\textsuperscript{*}\> Artificial Intelligence\textsuperscript{*}  \\
Probability Theory and Stochastic Processes\textsuperscript{*} \> \> Machine Learning\textsuperscript{\#} \\ 
Algorithms\textsuperscript{\#} \> Cryptoraphy\textsuperscript{\#} 
\end{tabbing}
\vspace{-0.5cm}
\hspace{9cm} \footnotesize{*Pursuing in spring 2014 \ \
\textsuperscript{\#}Online Courses at coursera.org
}

%%%%%%%%%%%%%%%%%%%%%%%%%%%%%%
\resheading{Projects and Papers}
%%%%%%%%%%%%%%%%%%%%%%%%%%%%%%

\begin{itemize}

\item {\large \textbf{\href{https://github.com/knsn1994/HitchHiker-s-Rover--GAME-}{3D bike race game} } \hspace{7.2cm} Prof. Subodh Kumar, 2012}\\ 
-Used OpenGL as graphics library. \\
-A dedicated physics engine was programmed. \\
-A database for High Scores is maintained. \\
-Used frustum culling for enhanchning the game speed. \\[5pt]

\item {\large \textbf{\href{https://github.com/knsn1994/processor}{Single Cycle Processor Design} } \hspace{3.5cm} Prof. Smruti Ranjan Sarangi, 2012} \\ 
-A single cycle risc processor was designed using \textbf{VHDL}\\ 
-The design implements simpleRisc ISA which is much simliar to ARM ISA \\[5pt]

\item {\large \textbf{\href{https://github.com/knsn1994/assembly_emulator}{Assembly Emulator} } \hspace{5.6cm} Prof. Smruti Ranjan Sarangi, 2012} \\ 
-Supports ARM assembly and SimpleRisc Assembly codes. \\ 
-Supports free flow of the code. \\
-Python's high order techniques like lambda functions, maps and filters used for programming the emulator. \\[5pt]

\item {\large \textbf{\href{http://www.cse.iitd.ernet.in/~cs1120231/walsh.pdf}{Walsh Codes, PN Sequences and their role in CDMA Technology} } }\\
-The paper outlines the basic of walsh codes and explores the mathematical aspect of Walsh codes as Finite Vector spaces. Also, it discuss properties and implementation of various PN sequences and then describe their role in CDMA Technology. \\[5pt]

\item {\large \textbf{\href{https://github.com/knsn1994/SocialNetworkSimulator}{Social Network Simulation and Analysis}} \hspace{3cm} Prof. Subodh Kumar, 2012}\\ 
-Multiprocessing and multithread used to achieve simulation. \\ 
-Inter process communication is eastablished by {\bf Message Queues}\\
-Final network after the simulation is stored as a graphml file. \\
-In the analysis of  the network, various queries such as shortest path, importance and clique size can be performed.\\[-5pt]
\end{itemize}


%%%%%%%%%%%%%%%%%%%%%%%%%%%%%%
\resheading{Sport Programming}
%%%%%%%%%%%%%%%%%%%%%%%%%%%%%%

\begin{itemize}

\item Indian team member in \textbf{International Olympiad in Informatics, 2012}
\item My team (team Angle) \textbf{ranked 9th in ICPC Amritapuri Regionals.} Best in IIT Delhi. 
\item TopCoder: \{rating: 1531, handle: \href{http://community.topcoder.com/tc?module=MemberProfile&cr=22917652}{knsn1994}\}; Codeforces: \{rating: 1740, handle: \href{http://codeforces.com/profile/knsn}{knsn}\}; \\[-5pt]
\end{itemize}

%%%%%%%%%%%%%%%%%%%%%%%%%%%%%%
\resheading{Awards, Grants \& Honours}
%%%%%%%%%%%%%%%%%%%%%%%%%%%%%%
\begin{tabular*}{7in}{l@{\extracolsep{\fill}}r}
 \textbf{International Olympiad in Informatics 2012}  & \textsc{July 2012} \\
Indian Team Member \\\\
 \textbf{\href{http://www.ipho2012.ee/silver-medalists/}{International Physics Olympiad 2012}}  & \textsc{July 2012} \\
Silver Award\\\\
 \textbf{\href{http://www.apho2012india.org/apho-2012/results}{Asian Physics Olympiad 2012}} & \textsc{May 2012} \\
Silver Award\\\\
 \textbf{\href{http://olympiads.hbcse.tifr.res.in/subjects/mathematics/inmo-2011-awardees/at_download/file}{Indian National Mathematics Olympiad 2011}} & \textsc{2011} \\
All India First Position \\\\
 \textbf{Institute Silver Medal for Academic Excellenece } & \textsc{Feb 2013} \\
Given for obtaining highest CGPA at IIT Delhi\\\\
 \textbf{\href{http://jee.iitd.ac.in/}{IIT-JEE 2012} All India Rank 18} & \textsc{May 2012} \\
Got 18th rank among more than 500,000 students\\\\
 \textbf{\href{http://www.adityabirlascholars.net/abgs_circle/2012.aspx}{Aditya Birla Scholarship 2012}} & \textsc{Sep 2012} \\
Given only to 15 engineering students all India\\\\
 \textbf{\href{http://www.opjems.com/opjems_scholars.aspx}{OP Jindal Engineering and Management Scholarship 2012}} & \textsc{Sep 2012} \\
Given to one student of each Year\\\\
 \textbf{\href{http://www.kvpy.org.in/main/}{KVPY} Scholarship 2011} & \textsc{2011} \\
Scholarship given for encouraging schience students \\\\
 \textbf{\href{http://aieee.nic.in/aieee2012/aieee/welcome.html}{AIEEE} 2012 All India Rank 19} & \textsc{May 2012} \\
Got 19th position among over 1 million students.
\end{tabular*}



%%%%%%%%%%%%%%%%%%%%%%%%%%%%%%
\resheading{Designing and Coding Skills}
%%%%%%%%%%%%%%%%%%%%%%%%%%%%%%

\begin{tabular*}{5in}{l@{\extracolsep{\fill}}l}
{\bf Extensive}
& \textsc{python}, C++\\[2pt]
{\bf Intermediate}
& \textsc{html5}, \textsc{css3}, \textsc{JavaScript, C, php}, \LaTeX, \textsc{Java, sql, bash, numpy}\\[2pt]
{\bf Basic}
& \textsc{Django Framework, Apple Script, Standard ML, BASIC, Perl} \\[-5pt]
\end{tabular*}

%%%%%%%%%%%%%%%%%%%%%%%%%%%%%%
\resheading{Community Involvement}
%%%%%%%%%%%%%%%%%%%%%%%%%%%%%%

\begin{itemize} 
\item Representative at \href{http://www.cse.iitd.ernet.in/~aces/}{ACES (Association of Computer Engineers, IIT Delhi)}
\item Member of Coding Club, IIT Delhi
\item Member of Electronics Club, IIT Delhi
\item Class Convener at IIT Delhi
\end{itemize}

%%%%%%%%%%%%%%%%%%%%%%%%%%%%%%
\resheading{Other Interests}
%%%%%%%%%%%%%%%%%%%%%%%%%%%%%%

\begin{itemize}
\item I have keen interest in human psychology and philosophy. I also write a blog: \href{https://tenet.quora.com}{Tenet}.
\item I love to play Guitar and Synth. 
\item I am a part of my hostel Dance Club. 
\end{itemize}

\end{document}
